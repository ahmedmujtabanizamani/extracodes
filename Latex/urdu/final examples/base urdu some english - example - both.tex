\documentclass[12pt]{article}

\usepackage{polyglossia}
%\usepackage{fontspec}

\setmainlanguage{urdu} % the document is in urdu
\setotherlanguage{english}   % with some bits in English
 % the following line specifies Urdu font and its location
%\setmainfont{Jameel Noori Nastaleeq}

\newfontfamily\urdufont[Script=Arabic]{Jameel Noori Nastaleeq}
\newfontfamily\urdufontsf[Script=Arabic]{Noto Sans Arabic}
%\renewcommand{\familydefault}{\rmdefault}
%set font for other language
\newfontfamily\englishfont{Arial}


\begin{document}
\title{\textenglish{Introduction to Urdu Poetry}}
\author{\textenglish{Roshan Dil}}
\maketitle
\tableofcontents


\section{\textenglish{English Text}}
\begin{otherlanguage}{english}
A document mainly in English has Urdu words 
(\texturdu{ جملے کے درمیان میں اردو الفاظ}) within a sentence.
But  also has Urdu paragraphs like below,
\end{otherlanguage}

\section{\textenglish{Urdu Text}}

اردو کے مشہور شعرا میں میرتقی میر، غالب،جوش ملیح ٓابادی، شکیل بدایونی،
 اقبال اور فیض کا نام سر فہرست آتا ہے۔  اردو کے بارے میں اردو کا  ایک مشہور شعر
\begin{center}
اردو ہے جس کا نام ، ہمیں جانتے ہیں داغ \\
سارے جہاں میں دھوم ہماری زباں کی ہے\\
\hspace{6cm} (داغ دھلوی)
\end{center}

\begin{english}
Returning  back to our writing in English.
\end{english}

\end{document}